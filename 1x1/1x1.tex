\documentclass[12pt]{article}
\usepackage[margin=0.7in]{geometry} 		% defines page margin
\usepackage{knitting} 				% defines \chart and \textknit
\usepackage{titling} 				% title page
\usepackage{graphicx,xspace, scrextend}	% defines space control stuff
\usepackage{tabularx, array, colortbl}	% defines tables
\usepackage{multicol} 				% defines columns
\usepackage{multirow} 				% defines multirows, combined cells in tables
\usepackage{framed} 				% defines boxes for notes and written directions
\usepackage[x11names]{xcolor} 		% extends color library
\pdfmapfile{+knitfont.map}

% font selection
\usepackage{palatino, moresize, sectsty}
\allsectionsfont{\sffamily}

\renewcommand{\arraystretch}{2} % compresses tables for pattern keys

\newcolumntype{L}[1]{>{\leftalign\arraybackslash}p{#1}}
\newcolumntype{C}[1]{>{\centering\arraybackslash}p{#1}}

% length parameters
\setlength{\parindent}{0pt} % disables indentation for paragraphs
\setlength{\columnsep}{0.7cm} % column separation in multicol environment

% color parameters
\colorlet{framecolor}{black}
\colorlet{shadecolor}{LemonChiffon1}
\colorlet{highlight}{yellow}

% custom commands
\newcommand{\comment}[1]{} % allows for multiline comments that LaTeX will ignore

\newcommand{\vocab}[1]{\emph{\textbf{#1}}} % format for highlighting definitions of stitches, vocabulary terms
\newcommand{\rowDir}[1]{\textbf{#1:}} % indent for written instructions within paragraphs

\renewcommand{\repeat}[1]{\textbf{*[#1]*}} % format for written repeats, bold with *[ stitches ]*
\newcommand{\x}{$\times$}			% times symbol but shorthand
\newcommand{\setrepeat}[2]{\textbf{[#1]}\x{#2}}		% format for repeats with set number of repetitions, bold with [ stitches ]

\newcommand{\blank}{\underline{\hspace{2em}} } % written instructions, fill-in-the-blank box
\newcommand{\highlighted}[1]{\colorbox{highlight}{#1}} % written instructions, highlight particular text


% stitch count commands
\newcommand{\increase}[1]{(\emph{+#1 
	\ifnum#1=1{st}\else{sts}\fi})}
\newcommand{\decrease}[1]{(\emph{$-$#1
	\ifnum#1=1{st}\else{sts}\fi})}
\newcommand{\stitchcount}[1]{(\emph{#1 sts})}

% marker instructions
\renewcommand{\pm}[1]{\emph{pm #1}} % place stitch marker
\newcommand{\sm}{\emph{sm}} % slip marker
\renewcommand{\rm}[1]{\emph{rm #1}} % remove stitch marker

% thick horizontal line
\makeatletter \newcommand{\thickhline}{
    \noalign {\ifnum 0=`}\fi \hrule height 1.5pt
    \futurelet \reserved@a \@xhline
}
\makeatother

% custom environments
\newenvironment{frnote}
    {% framed environment for pattern notes
    	\setlength{\FrameRule}{1.5pt}
    	\def\FrameCommand{\fboxrule=\FrameRule\fboxsep=\FrameSep \fcolorbox{framecolor}{shadecolor}}
    	\MakeFramed {\FrameRestore}}
    {\setlength{\FrameRule}{1pt}
	\endMakeFramed}

\newenvironment{frdirection}
    {% framed environment for written directions
	\def\FrameCommand{\fboxrule=\FrameRule\fboxsep=\FrameSep \fbox}
   	\MakeFramed {\advance\hsize-\width \FrameRestore}
    	\addmargin[1.5cm]{0pt}}
    {\endaddmargin
	\endMakeFramed}

\newenvironment{unframed}
    {% unframed environment for written directions
	\begin{addmargin}[2em]{0pt}
	\setlength{\parindent}{-2em}}
    {%\vspace{1em}
	\setlength{\parindent}{0em}
	\end{addmargin}}

\title{One By One} % pattern name here
\author{Shanel Wu (Piper Nell)}

\begin{document}

%%%%%%%%%%%%%%%%%%%%%%%%%%%%%%%%%%%%%%%%%%%%%%%%%%
% TITLE PAGE 
\begin{titlingpage}

% COVER PHOTO
% uncommend line below if you want a background fill image
% \ThisLRCornerWallPaper{1.0}{image.jpg} 

{\fontfamily{qag}\selectfont
\HUGE\textbf{\thetitle}
\hspace{2em} % adjust this space
\normalsize\theauthor
}

\begin{multicols}{2}

Nothin' but a long, foldable 1x1 ribbed beanie! This pattern is written for fingering (DK, worsted) weight yarn.

\subsection*{Yarn Requirements}

% yardage, number of colors, etc.

% also include: sample yarn, other yarn suggestions

\subsection*{Tools}

\begin{itemize}
\item 32" or longer circular needle in Size A for cast on
\item 16" circular needle in Size A for body
\item tapestry needle
\item % STITCH MARKERS
\end{itemize}

\subsection*{Sizing/Gauge}

% sample measurements, gauge, notes on ease, etc.

\subsection*{Techniques}

This pattern is suitable for an % DIFFICULTY LEVEL
Prior to knitting this pattern, you should be familiar with % PREREQUISITE TECHNIQUES
For a complete list of stitches used, see Pattern Key.

% discuss any special techniques and tutorials included

\vfill
\columnbreak

\subsection*{Pattern Key}

% stitch key - fill in with all stitches used in design: chart symbol, written abbreviation, full stitch name or explanation
% stitches with explanations must first be BOLDED, followed by colon then explanation
\begin{center}
{\renewcommand{\arraystretch}{1.5}
\begin{tabular}{| C{0.2\linewidth}  p{0.6\linewidth} | }
\thickhline \rowcolor{shadecolor} 
\textbf{Abbr.}	& \textbf{Description} \\ \thickhline
k	&  knit \\
p	& purl   \\
k tbl	& knit through the back loop \\
yo		& yarn-over  \\
k2tog 	& knit 2 together \\
ssk		& slip slip knit \\
k3tog 	& knit 3 together \\
sssk		& \textbf{slip slip slip knit:} slip 3 knitwise individually, k3tog through back loop\\
sk2p	& \textbf{slip k2tog psso} slip 1 st knitwise, k2tog, pass slipped st over (psso)\\
\hline
\end{tabular}
}
\end{center}
\end{multicols}
\end{titlingpage}

%%%%%%%%%%%%%%%%%%%%%%%%%%%%%%%%%%%%%%%%%%%%%%%%%%
% BEGIN INSTRUCTIONS

\subsection*{Cast On}

This section walks you through a tubular cast on using Judy's Magic Cast On. If you wish, you may CO the same number of sts using another method.

With 

\subsection*{Body}

Work 1x1 ribbing as follows until hat measures 12" long.

\subsection*{Crown Decreases and Finishing}

%%%%%%%%%%%%%%%%%%%%%%%%%%%%%%%%%%%%%%%%%%%%%%%%%%
% APPENDICES (IF ANY)

\end{document}