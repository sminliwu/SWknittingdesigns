\documentclass[12pt]{article}
\usepackage[margin=0.7in]{geometry} 		% defines page margin
\usepackage{knitting} 				% defines \chart and \textknit
\usepackage{titling} 				% title page
\usepackage{graphicx,xspace, scrextend}	% defines space control stuff
\usepackage{tabularx, array, colortbl}	% defines tables
\usepackage{multicol} 				% defines columns
\usepackage{multirow} 				% defines multirows, combined cells in tables
\usepackage{framed} 				% defines boxes for notes and written directions
\usepackage[x11names]{xcolor} 		% extends color library
\pdfmapfile{+knitfont.map}

% font selection
\usepackage{palatino, moresize, sectsty}
\allsectionsfont{\sffamily}

\renewcommand{\arraystretch}{2} % compresses tables for pattern keys

\newcolumntype{L}[1]{>{\leftalign\arraybackslash}p{#1}}
\newcolumntype{C}[1]{>{\centering\arraybackslash}p{#1}}

% length parameters
\setlength{\parindent}{0pt} % disables indentation for paragraphs
\setlength{\columnsep}{0.7cm} % column separation in multicol environment

% color parameters
\colorlet{framecolor}{black}
\colorlet{shadecolor}{LemonChiffon1}
\colorlet{highlight}{yellow}

% custom commands
\newcommand{\comment}[1]{} % allows for multiline comments that LaTeX will ignore

\newcommand{\vocab}[1]{\emph{\textbf{#1}}} % format for highlighting definitions of stitches, vocabulary terms
\newcommand{\rowDir}[1]{\textbf{#1:}} % indent for written instructions within paragraphs

\renewcommand{\repeat}[1]{\textbf{*[#1]*}} % format for written repeats, bold with *[ stitches ]*
\newcommand{\x}{$\times$}			% times symbol but shorthand
\newcommand{\setrepeat}[2]{\textbf{[#1]}\x{#2}}		% format for repeats with set number of repetitions, bold with [ stitches ]

\newcommand{\blank}{\underline{\hspace{2em}} } % written instructions, fill-in-the-blank box
\newcommand{\highlighted}[1]{\colorbox{highlight}{#1}} % written instructions, highlight particular text


% stitch count commands
\newcommand{\increase}[1]{(\emph{+#1 
	\ifnum#1=1{st}\else{sts}\fi})}
\newcommand{\decrease}[1]{(\emph{$-$#1
	\ifnum#1=1{st}\else{sts}\fi})}
\newcommand{\stitchcount}[1]{(\emph{#1 sts})}

% marker instructions
\renewcommand{\pm}[1]{\emph{pm #1}} % place stitch marker
\newcommand{\sm}{\emph{sm}} % slip marker
\renewcommand{\rm}[1]{\emph{rm #1}} % remove stitch marker

% thick horizontal line
\makeatletter \newcommand{\thickhline}{
    \noalign {\ifnum 0=`}\fi \hrule height 1.5pt
    \futurelet \reserved@a \@xhline
}
\makeatother

% custom environments
\newenvironment{frnote}
    {% framed environment for pattern notes
    	\setlength{\FrameRule}{1.5pt}
    	\def\FrameCommand{\fboxrule=\FrameRule\fboxsep=\FrameSep \fcolorbox{framecolor}{shadecolor}}
    	\MakeFramed {\FrameRestore}}
    {\setlength{\FrameRule}{1pt}
	\endMakeFramed}

\newenvironment{frdirection}
    {% framed environment for written directions
	\def\FrameCommand{\fboxrule=\FrameRule\fboxsep=\FrameSep \fbox}
   	\MakeFramed {\advance\hsize-\width \FrameRestore}
    	\addmargin[1.5cm]{0pt}}
    {\endaddmargin
	\endMakeFramed}

\newenvironment{unframed}
    {% unframed environment for written directions
	\begin{addmargin}[2em]{0pt}
	\setlength{\parindent}{-2em}}
    {%\vspace{1em}
	\setlength{\parindent}{0em}
	\end{addmargin}}

\title{German Short Row (GSR) Heel} % pattern name here
\author{Shanel Wu (Piper Nell)}

\begin{document}

%%%%%%%%%%%%%%%%%%%%%%%%%%%%%%%%%%%%%%%%%%%%%%%%%%

{\fontfamily{qag}\selectfont
\HUGE\textbf{\thetitle}
}

\vspace{1em}
This heel pattern will work with a cuff-down or toe-up sock. Set up by dividing stitches in half. Heel is worked back-and-forth across one half of the stitches.

\begin{multicols}{2}
%%%%%%%%%%%%%%%%%%%%%%%%%%%%%%%%%%%%%%%%%%%%%%%%%%
% BEGIN INSTRUCTIONS
\section*{Part 1}
\begin{unframed}
\rowDir{Row 1} k to 1 st before end of half, dst \\
\rowDir{Row 2} p to 1 st before end of half, dst \\
\rowDir{Row 3} k to previous dst, dst \\
\rowDir{Row 4} p to previous dst, dst \\
\end{unframed}

Repeat Rows 3 and 4 until you have roughly a third of your sts are double sts. (e.g. for a 30 st heel, you will have 9 double sts on each side and 10 sts in the middle. For a 20 st heel, you will have 6 double sts on each side and 6 sts in the middle. For any size heel, you will have one regular st on each end.)

\vspace{-1em}

\section*{Boomerang}
\begin{unframed}
\rowDir{Row 1} k to dst, work all double sts as they come, k1 and dst \\
\rowDir{Row 2} p to dst, work all double sts as they come, p1 and dst \\
\end{unframed} 

\vspace{-1em}

\section*{Part 2}
\begin{unframed}
\rowDir{Row 1} k to \blank sts before end of needle, dst \\
\rowDir{Row 2} p to \blank sts before end of needle, dst \\
\rowDir{Row 3} k to previous dst, work dst, k1 and dst \\
\rowDir{Row 4} p to previous dst, work dst, k1 and dst
\end{unframed} \vspace{1em}

Repeat Rows 3 and 4 until you've worked a double st at the 2nd to last st on both sides. On the next knit row, k to dst, work 2 double sts. Return to working in the round. You will have 2 unworked double sts on the other end of your needle. Work across the instep, then work the two double sts. Your heel is done!

\vfill
\columnbreak

\begin{frnote} \vspace{-1em}
\subsubsection*{Double Stitch (dst/GSR turn)}
\rowDir{k side} turn work to WS, bring yarn forward, slip 1 as if to purl. Pull yarn up, over, and behind the needle to create a \vocab{double st}, and bring yarn forward to purl.

\rowDir{p side} turn work to RS, bring yarn forward, slip 1 as if to purl. Pull yarn up, over, and behind the needle to create a double st, and keep yarn in back.
\end{frnote}

\vspace{2em}

\begin{frnote} \vspace{-1em}
\subsubsection*{Working a double st}
\rowDir{k side} insert needle into both loops of the double st as if to knit, k2tog.

\rowDir{p side} insert needle into both loops of the double st as if to purl, p2tog.
\end{frnote}

\begin{frnote} \vspace{-1em}
\subsubsection*{Filling in the Blank}
The number of sts in Part 2 depends on your total stitch count. Remember how many double sts you had at the end of Part 1 and add 1 to fill in the blank. (e.g. If you had 9 double sts on each side, you will write 10 in the blank.)
\end{frnote}

\end{multicols}

%%%%%%%%%%%%%%%%%%%%%%%%%%%%%%%%%%%%%%%%%%%%%%%%%%
% APPENDICES (IF ANY)

\end{document}