\documentclass[12pt]{article}
\usepackage[margin=0.7in]{geometry} 		% defines page margin
\usepackage{knitting} 				% defines \chart and \textknit
\usepackage{titling} 				% title page
\usepackage{graphicx,xspace, scrextend}	% defines space control stuff
\usepackage{tabularx, array, colortbl}	% defines tables
\usepackage{multicol} 				% defines columns
\usepackage{multirow} 				% defines multirows, combined cells in tables
\usepackage{framed} 				% defines boxes for notes and written directions
\usepackage[x11names]{xcolor} 		% extends color library
\pdfmapfile{+knitfont.map}

\renewcommand{\arraystretch}{2}

\newcolumntype{L}[1]{>{\leftalign\arraybackslash}p{#1}}
\newcolumntype{C}[1]{>{\centering\arraybackslash}p{#1}}

% length parameters
\setlength{\parindent}{0pt} % disables indentation for paragraphs
\setlength{\columnsep}{0.7cm} % column separation in multicol environment

% color parameters
\colorlet{framecolor}{black}
\colorlet{shadecolor}{LemonChiffon1}
\colorlet{highlight}{yellow}

% custom commands
\newcommand{\comment}[1]{} % allows for multiline comments that LaTeX will ignore

\newcommand{\vocab}[1]{\emph{\textbf{#1}}} % format for highlighting definitions of stitches, vocabulary terms
\newcommand{\rowDir}[1]{\textbf{#1:}} % indent for written instructions within paragraphs

\renewcommand{\repeat}[1]{\textbf{*[#1]*}} % format for written repeats, bold with *[ stitches ]*
\newcommand{\setrepeat}[1]{\textbf{[#1]}}		% format for repeats with set number of repetitions, bold with [ stitches ]

\newcommand{\increase}[1]{(\emph{+#1 
	\ifnum#1=1{st}\else{sts}\fi})}
\newcommand{\decrease}[1]{(\emph{$-$#1
	\ifnum#1=1{st}\else{sts}\fi})}
\newcommand{\stitchcount}[1]{(\emph{#1 sts})}

\newcommand{\blank}{\underline{\hspace{2em}} }

\newcommand{\spine}[1]{\colorbox{highlight}{#1}}

\renewcommand{\pm}{\vocab{pm}} % place stitch marker
\newcommand{\sm}{\vocab{sm}} % slip marker
\renewcommand{\rm}{\vocab{rm}} % remove stitch marker

% thick horizontal line
\makeatletter \newcommand{\thickhline}{
    \noalign {\ifnum 0=`}\fi \hrule height 1.5pt
    \futurelet \reserved@a \@xhline
}
\makeatother

% custom environments
\newenvironment{frnote}
    {% framed environment for pattern notes
    	\setlength{\FrameRule}{1.5pt}
    	\def\FrameCommand{\fboxrule=\FrameRule\fboxsep=\FrameSep \fcolorbox{framecolor}{shadecolor}}
    	\MakeFramed {\FrameRestore}}
    {\setlength{\FrameRule}{1pt}
	\endMakeFramed}

\newenvironment{frdirection}
    {% framed environment for written directions
	\def\FrameCommand{\fboxrule=\FrameRule\fboxsep=\FrameSep \fbox}
   	\MakeFramed {\advance\hsize-\width \FrameRestore}
    	\addmargin[1.5cm]{0pt}}
    {\endaddmargin
	\endMakeFramed}

\newenvironment{unframed}
    {% unframed environment for written directions
	\begin{addmargin}[2em]{0pt}
	\small
	\setlength{\parindent}{-2em}}
    {\vspace{1em}
	\normalsize
	\end{addmargin}
	\setlength{\parindent}{0em}}

\title{Whims of the Wind Shawl}
\author{Shanel Wu (Piper Nell)}

\begin{document}
\begin{titlingpage}
\begin{multicols}{2}

% COVER PHOTO

\section*{\thetitle}
\vspace{-0.5em}
\subsubsection*{\theauthor}

\subsection*{Yarn Requirements}

At least 400y of yarn in any weight.

\subsection*{Tools}

\begin{itemize}
\item 32"/100cm or longer circular needle in size appropriate for yarn weight
\item tapestry needle
\item 1 stitch marker
\end{itemize}

\subsection*{Sample Specifications}

Sample A -- needles: US6/4.0mm 40" circular needle -- yarn: Knit Picks Andean Treasure (110y/50g, heavy sport weight) in Finnley (off-white), Calypso (blue-green), Sapphire (medium blue), and Mystery (dark gray).

\subsection*{Techniques}

This pattern is suitable for an advanced beginner. Prior to knitting this pattern, you should be familiar with knits, purls, and basic increases (yo and kfb). For a complete list of stitches used, see Pattern Key.

\vspace{1em}
Double yarn-overs (yoyo) and working into them are crucial techniques in this pattern. For additional support, refer to the tutorial at the end of the pattern.

\subsection*{Pattern Key}


\begin{center}
\begin{tabular}{| C{0.12\linewidth}  C{0.2\linewidth}  p{0.6\linewidth} | }
\thickhline \rowcolor{shadecolor} 
\textbf{Chart}	& \textbf{Written}	& \textbf{Description} \\ \thickhline
\chart{-}	& k (RS); p~(WS)	&  knit (RS); purl (WS)	\\
\chart{=} 	& p (RS); k~(WS)	& purl (RS); knit (WS)  \\
\chart{O} 	& yo		& yarn-over  \\
\chart{OO}	& yoyo	& double yarn-over \\
\chart{>}	& k2tog 	& knit 2 together \\
{}		& kfb		& knit front back: knit in the front loop as usual, then k tbl in the same stitch \\
\chart{-t} (WS)	& pfb		& purl front back: purl in the front loop as usual, then purl tbl in the same stitch\\
{}		& w\&t 	& wrap \& turn: s1 purlwise wyib, bring yarn forward, slip st back to l needle, turn (the slipped st is ``wrapped" by the working yarn) \\
% ADD STITCHES AS NEEDED
\hline
\end{tabular}
\end{center}
\end{multicols}

\end{titlingpage}

\begin{multicols}{2}
\section*{0. Color Management}
% color management
In this pattern, I leave the color changes up to you. Whether you use one color, a palette of 3-5 coordinating colors, or an eclectic mix of scraps, use your colors in the given shaping methods to create a unique shawl. Always change colors at the beginning of a RS row.

% striping tips

% choosing textures

\subsection*{1. Cast On and Set Up}

Using the long-tail method, CO 6 sts. Turn and work the following rows:
\begin{unframed}
\rowDir{Row 1} (RS) k2, \spine{yo, k2tog,} \pm, k2 \stitchcount{6} \\
\rowDir{Row 2} (WS) k2, \spine{\sm, p2,} k2  \\
\rowDir{Row 3} k2, \spine{yo, k2tog, \sm, yo,} k2 \stitchcount{7} \\
\rowDir{Row 4} k3, \spine{\sm, p2,} k2 \\
\rowDir{Row 5} k2, \spine{yo, k2tog, \sm,} k3 \stitchcount{7} \\
\rowDir{Row 6} Repeat Row 4. \\
\rowDir{Row 7} k2, \spine{yo, k2tog, \sm, yo,} k3 \stitchcount{8} \\
\rowDir{Row 8} k4, \spine{\sm, p2,} k2 \\
\rowDir{Row 9} k2, \spine{yo, k2tog, \sm,} k4 \stitchcount{8} \\
\rowDir{Row 10} Repeat Row 8. \\
\rowDir{Row 11} k2, \spine{yoyo, k2tog, \sm, yo,} k4 \stitchcount{10} \\
\rowDir{Row 12} k5, \spine{\sm, p1, pfb,} k2 \\
\end{unframed}

\chart[evenleft]{

=====--t==
----O>OO--
====--==
---->O--
====--==
---O>O--
===--==
--->O--
===--==
--O>O--
==--==
-->O--
}

\vfill
\columnbreak
\subsection*{2. Kite Section: $\sim$50\% of yarn}
The shaping of this section takes place over 4-row repeats, with a 2-stitch garter edging and increases taking place on both sides of a central spine. General formula:
\begin{unframed}
\rowDir{Row 1} k2, work to 2 sts from m, \spine{yo, k2tog, \sm,} work to 2 sts from end, k2\\
\rowDir{Row 2} k2, work to marker, \spine{\sm,} \spine{p1, work 1 st,} work to 2 sts from end, k2\\
\rowDir{Row 3} k2, work to 2 sts from m, \spine{yoyo, k2tog,} \spine{\sm, yo,} work to 2 sts from end, k2 \\
\rowDir{Row 4} k2, work to m, \spine{\sm, p1, work kfb or pfb,} work to 2 sts from end, k2
\end{unframed}

I've provided three texture variations for this section. Work your choice of texture until section is desired size.

\emph{Stockinette variation:}
\begin{unframed}
\rowDir{Row 1} (RS) k to 2 sts from marker, \spine{yo, k2tog,} \sm, k to end \\
\rowDir{Row 2} (WS) k2, p to marker, \sm, \spine{p2,} p to 2 sts from end, k2 \\
\rowDir{Row 3} k to 2 sts from marker, \spine{yoyo, k2tog,} \sm, yo, k to end \increase{2} \\
\rowDir{Row 4} k2, p to marker, \sm, \spine{p1, pfb,} p to 2 sts from end, k2
\end{unframed}

\emph{Garter variation:}
\begin{unframed}
\rowDir{Row 1} (RS) k to 2 sts from marker, \spine{yo, k2tog,} \sm, k to end \\
\rowDir{Row 2} (WS) k to marker, \sm, \spine{p1, k1,} k to end \\
\rowDir{Row 3} k to 2 sts from marker, \spine{yoyo, k2tog,} \sm, yo, k to end \increase{2} \\
\rowDir{Row 4} k to marker, \sm, \spine{p1, kfb,} k to end
\end{unframed}

\emph{Garter ridges variation:}
\begin{unframed}
\rowDir{Row 1} (RS) k to 2 sts from marker, \spine{yo, k2tog,} \sm, k to end \\
\rowDir{Row 2} (WS) k2, p to marker, \sm, \spine{p2,} p to 2 sts from end, k2 \\
\rowDir{Row 3} k to 2 sts from marker, \spine{yoyo, k2tog,} \sm, yo, k to end \increase{2} \\
\rowDir{Row 4} k to marker, \sm, \spine{p1, kfb,} k to end
\end{unframed}

\subsection*{3. Garter Wedge: $\sim$25\% of yarn}

In this section, you will use wrap-and-turn (w\&t) short rows to create a triangular wedge, turning the piece into an asymmetrical triangle. Set up as follows:
\begin{unframed}
\rowDir{Set Up 1} (RS) k1, kfb, k1, w\&t \\
\rowDir{Set Up 2} (WS) k to end
\end{unframed}

Repeat Rows 1 and 2 until there are 2 unworked sts before the marker (wrapped st will be 3rd from marker). 
\begin{unframed}
\rowDir{Row 1} (RS) k1, kfb, k to wrapped st, work wrapped st by inserting needle knitwise into wrap then k together with st, k1, w\&t \\
\rowDir{Row 2} (WS) k to end
\end{unframed}
Work Rows 3 and 4 once.
\begin{unframed}
\rowDir{Row 3} (RS) k1, kfb, k to wrapped st, work wrapped st, k2, \rm, k to end of row \\
\rowDir{Row 4} (WS) k to end
\end{unframed}

\vfill
\columnbreak

\subsection*{4. Sail Section: remainder of yarn}

Your shawl is now an asymmetrical triangle. The shaping of this section takes place over 2-row repeats, with a 2-stitch garter edge and increases taking place on one edge only. General formula:
\begin{unframed}
\rowDir{Row 1} k1, kfb twice, work to 2 sts from end, k2 \\
\rowDir{Row 2} k2, work to 2 sts from end, k2
\end{unframed}

Work your choice of texture until shawl is desired size or until yarn is used up.

\emph{Stockinette variation:}
\begin{unframed}
\rowDir{Row 1} k1, kfb twice, k to end \\
\rowDir{Row 2} k2, p to 2 sts from end, k2
\end{unframed}

\emph{Garter variation:}
\begin{unframed}
\rowDir{Row 1} k1, kfb twice, k to end \\
\rowDir{Row 2} k to end
\end{unframed}

\emph{Ribbing variation (requires odd stitch count):}
\begin{unframed}
\rowDir{Row 1} k1, kfb, \repeat{k1, p1} to 3 sts from end, k3 \\
\rowDir{Row 2} k2, \repeat{p1, k1} to 3 sts from end, p1, k2 
\end{unframed}

Bind off loosely and block. Enjoy your shawl!
\end{multicols}

\end{document}